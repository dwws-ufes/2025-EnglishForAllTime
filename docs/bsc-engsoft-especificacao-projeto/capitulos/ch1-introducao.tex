% ==============================================================================
% Projeto de Sistema - Matheus De Oliveira Lima
% Capítulo 1 - Introdução
% ==============================================================================
\chapter{Introdução}
\label{sec-intro}
\vspace{-1cm}

Este documento apresenta o projeto (\textit{design}) do sistema \emph{\imprimirtitulo}. 

É um plataforma de ensino de inglês onde o professor-administrador tem controle total: ele pode cadastrar seus alunos e adicionar/criar cursos diretamente no sistema, organizando-os em módulos com vídeos (via links do YouTube não listados), materiais em PDF e exercícios. A plataforma oferece um painel intuitivo para o dono gerenciar tanto os usuários quanto os conteúdos publicados, permitindo atualizações rápidas e personalizadas, sem depender de terceiros. E para os alunos eles terão uma página com todos os conteúdos publicados pelo professor.

%\vitor{Completar o parágrafo acima com uma breve descrição do sistema.}

Além desta introdução, este documento está organizado da seguinte forma: 
a Seção~\ref{sec-plataforma} apresenta a plataforma de software utilizada na implementação do sistema;
a Seção~\ref{sec-rnfs} apresenta a especificação dos requisitos não funcionais (atributos de qualidade), definindo as táticas e o tratamento a serem dados aos atributos de qualidade considerados condutores da arquitetura; 
a Seção~\ref{sec-arquitetura} apresenta a arquitetura de software; por fim, 
a Seção~\ref{sec-frameweb} apresenta os modelos FrameWeb que descrevem os componentes da arquitetura.

